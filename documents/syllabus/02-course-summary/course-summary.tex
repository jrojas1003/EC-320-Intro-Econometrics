\section*{COURSE SUMMARY}

This course introduces statistical techniques that economists use to test economic theories and to estimate the relationships between economic variables. 
Econometrics combines economics and statistics with data to analyze and measure economic phenomena. 
In this course, we will focus our attention on regression analysis -- the workshorse of applied econometrics. 
Using calculus and introductory statistics, we will cultivate a working understanding of the theory underpinning regression analysis, emphasizing the assumptions we must make to make causal statements. 
Satistical programming is fundamental to practicing applied econometrics. 
Thus we will teach the statistical programming language \texttt{R} to apply insights from theory and learn how to work with data.
To the extent that you invest the requisite time and effort, you can leave this course with marketable skills in data analysis and -- most importantly -- a more sophisticated understanding of the notion that \textbf{correlation does not necessarily imply causation}.

\subsection*{SOFTWARE}
\vspace*{-0.5cm}
\rule{\textwidth}{2pt}

\begin{itemize}
    \item We will use the statistical programming language \texttt{R}
    \item We will use \texttt{RStudio} to interact with \texttt{R}
\end{itemize}

Learning \texttt{R} is challenging, but well worth the effort. 
\texttt{R} is a powerful and versatile tool for data analysis and visualization, which makes it popular amongst employers.
If you dedicate the time and effort necessary to learn the language, you are likely to reap a handsome return on the job market. 
I expect that you install \texttt{R} and \texttt{RStudio} on your own computer.
Do not worry, \textbf{both are free}. 
I also recommend that you be thoughtful of how you choose to organize your saved scripts, data, and assignments (e.g. Home > Documents > Classes > EC320).
For convenience, I make material available through the \hyperlink{https://jrojas1003.github.io/EC-320-Intro-Econometrics/}{Course Website}

\subsection*{TEXTBOOKS AND OTHER READINGS}
\vspace*{-0.5cm}
\rule{\textwidth}{2pt}
\vspace{0.1cm}

\textbf{Econometrics:}
\begin{itemize}
    \item Introduction to Econometrics, 5$^{th}$ Ed. by Christopher Dougherty (\textbf{ItE})
    \item Mastering 'Metrics: The Path from Cause to Effect by Angrist and Pischke (\textbf{MM})
\end{itemize}

You can purchase these through the UO duckstore or your preferred online bookseller. 
These books are a great primary source of knowledge. 
I recommend that you read (or at least skim) the assigned readings as you go along. 
The lectures and the readings are meant to \textbf{complement} one another. 

\vspace{0.75cm}

\textbf{R Books:}
\begin{itemize}
    \item \hyperlink{target name}{R for Data Science} by Hadley Wickham and Garrett Grolemund
    \item \hyperlink{target name}{Introduction to Econometrics with R} by Hanck, Arnold, Gerber, and Chmeizer
\end{itemize}

These are great open-source resources to help you navigate \texttt{R}.

\subsection*{PREREQUISITES}
\vspace*{-0.5cm}
\rule{\textwidth}{2pt}
\begin{itemize}
    \item Math 242 (Calculus)
    \item Math 243 (Introduction to Statistics) or equivalent
\end{itemize}

\section*{ASSIGNMENTS AND EXAMS}

\subsection*{PROBLEM SETS \& QUIZZES}
\vspace*{-0.5cm}
\rule{\textwidth}{2pt}

Every week, there will be \textbf{one problem set} as well as a \textbf{Canvas quiz} to complete.

\textbf{Problem sets} will primarily focus on analytical problems but may include a computational component. 
Submission \textbf{must be your own work}.
You will receive \textbf{zero points} for copied work or work generated by an AI tool (ChatGPT, Copilot, Gemini, etc.)

\begin{itemize}
    \item Due on Friday midnight every week
    \item \textbf{PDF} and \textbf{html} are the only file types accepted as submissions
    \item One file per problem set submission
    \item Your lowest problem set score will be dropped
\end{itemize}

Feel free to work together on assignments.
Unless explicitly stated, \textbf{each student is required to write and submit independent answers}.
Any suspicion of of copying or cheating will be viewed as academic dishonesty and be treated as such. 
In other words: you must place answers \textbf{in your own words and written in your own code}.
Copying from other people (even if you worked with them) or from previous assignments is considered cheating. 
Assignments will be submitted on Canvas under the "Assignments" tab. 

\textbf{Quizzes} will be short multiple choice or numeric questions to help you check your understanding as the course progresses. 
\begin{itemize}
    \item They will be graded based on completion only to make sure you are following along
\end{itemize}

Exams will be open-note, but you are not allowed to work with others. 
All work should be completed independently and in your own words and/or code.

\subsection*{LATE POLICY}
\vspace*{-0.5cm}
\rule{\textwidth}{2pt}

\begin{itemize}
    \item Late assignments will be accepted \textbf{up to 36 hours late} with a penalty of \textbf{2\% per hour late}.
    \item For example, when submitting 10 hours late, an assignment with a 90\% score would be penalized by 20\%, resulting in a final grade of 70\%.
\end{itemize}

\subsection*{EXAMS}
\vspace*{-0.5cm}
\rule{\textwidth}{2pt}
\begin{itemize}
    \item The \textbf{Midterm} will be a timed Canvas quiz which will be released \textbf{Thursday, July 03 at 12:00pm}. You will have 12 hours to complete it and upload your answers.
    \item The \textbf{Final} will be on \textbf{Friday, July 18th at 12:00pm}. You will have 12 hours to complete it and upload your answers.
    \item These are subject to change but I will give you ample notice if that is the case. 
\end{itemize}

\subsection*{MAKEUP ASSIGNMENTS}
\vspace*{-0.5cm}
\rule{\textwidth}{2pt}

There will be no makeup assignments. 
