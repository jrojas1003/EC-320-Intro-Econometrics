\documentclass[12pt]{exam}
\usepackage{amsthm}
\usepackage{libertine}
\usepackage[utf8]{inputenc}
\usepackage[margin=1in]{geometry}
\usepackage{amsmath,amssymb}
\usepackage{multicol}
\usepackage[shortlabels]{enumitem}
\usepackage{siunitx}
\usepackage{cancel}
\usepackage{graphicx}
\usepackage{listings}
\usepackage{tikz}
\usepackage[T1]{fontenc}  % Ensure proper encoding
\usepackage{titlesec}
\usepackage{booktabs}

\CorrectChoiceEmphasis{\color{red}\bfseries\itshape}

\newcommand{\class}{\fontfamily{lmss}\selectfont \textbf{EC 320}} % This is the name of the course 
\newcommand{\examnum}{\fontfamily{lmss}\selectfont \textbf{Problem Set 02}} % This is the name of the assignment
\newcommand{\examdate}{\textbf{Due 07/05 @ 11:59pm}} % This is the due date
\newcommand{\timelimit}{}
\addpoints % Allows to add points up to include table at end of document

\pagestyle{headandfoot} % Fancy equivalent for exam documentclass
\firstpageheadrule % Horizontal bar in first page
\runningheadrule % Horizontal bar in rest of pages
\firstpageheader{\class}{\examnum}{\fontfamily{lmss}\selectfont  \examdate} % 1st page header content
\firstpagefooter{}{Page \thepage}{} % 1st page footer content
\runningheader{\class}{\examnum}{\fontfamily{lmss}\selectfont  \examdate} % Rest of pages header content
%\runningfooter{Points earned: \makebox[1in]{\hrulefill} / \pointsonpage{\thepage} points}{}{\thepage\ of \numpages} % Rest of pages footer content
\bracketedpoints % Puts points per question inside brackets instead of parenthesis 

\titleformat{\section}
  {\normalfont\large\bfseries\sffamily}{\thesection}{1em}{}

\titleformat{\subsection}
  {\normalfont\normalsize\bfseries\sffamily}{\thesection}{1em}{}

\begin{document}
\fontfamily{lmss}\selectfont

We determined that (given some assumptions): $\hat{\beta}_{1}$ is distributed $\mathcal{N} \left(\beta_{1}, \dfrac{Var(u)}{\sum_{i} (x_{i} - \bar{x})^{2}}\right)$.
$Var(u)$ is unknown since $u_{i}$ is unobservable, so we have to approximate it using the regression residuals $e_{i}$.
Because of this, we also use the t-distribution, which is similar to the Normal distribution, but with slightly fatter tails. 

We define "standard errors" as our estimate of the standard deviation of the regression coefficient. 
The formula for a simple regression standard error of $\hat{\beta}_{1}$ is:

$$\sqrt{\dfrac{\sum_{i}(u_{i}^{2})}{(n-2)\sum_{i} (x_{i} - \bar{x})^{2}}}$$.

\begin{questions}


\question %01
\textbf{We would like our standard errors to be as small as possible so we can} increase the precision of our estimates. 
If we increased the number of observations (assuming all else is held equal), will our standard errors increase or decrease?

\vspace*{\stretch{1}}

\question %02
\textbf{All else held equal, if the sample variance of $x_{i}$ decreased, should we} expect the standard errors would increase or decrease?

For the sake of building intuition, we will use the following study:

There are two studies designed to find the effect of a blood pressure medication on health outcomes. 

Study A: 9 people take the placebo, 1 person takes the medication, so $X = (0,0,0,0,0,0,0,0,1)$

Study B: 5 people take the placebo, 5 people take the medication, so $X = (0,0,0,0,0,1,1,1,1,1)$.

Calculate the sample variance of $X$ in both studies (sample variance = $\dfrac{\sum_{i} (x_{i} - \bar{x})^{2}}{n - 1})$.
Which study will yield a more confident estimate of the effect, and which study has a lower $Var(X)$?

\vspace*{\stretch{1}}

\newpage 

\question %04
\textbf{Inference}

Say you run the following model and have the data below:

$$
    \text{Exam Score}_{i} = \beta_{0} + \beta_{1} \text{Hours Studied}_{i} + u_{i}
$$

\begin{table}[ht]
    \centering 
    \begin{tabular}{ccc}
        \toprule 
        Student & $X_{i}$ Hours Studied & $Y_{i}$ Exam Score \\
        \midrule
        1       &           2           &        65          \\
        2       &           3           &        70          \\
        3       &           5           &        75          \\
        4       &           6           &        78          \\
        5       &           8           &        85          \\       
        \bottomrule 
    \end{tabular}
\end{table}

You run the regression and receive the estimates: $\hat{\beta}_{0} = 59.35$ and $\hat{\beta}_{1} = 3.184$

\begin{parts}
    \part Recall we can estimate the variance of the error term $\sigma^{2}$ using $\dfrac{\sum_{i} \hat{u}_{i}}{n - k}$ where $n$ is the number of observations and $k$ are the number of regression parameters.   
    What is $\sigma^{2}$ for this model?
    \vspace*{\stretch{1}}
    \part Calculate the Standard Error of $\hat{\beta}_{1}$
    \vspace*{\stretch{1}}
    \part Find the 95\% Confidence Interval for $\hat{\beta}_{1}$ using the t-value of 3.182
    \vspace*{\stretch{1}}
\end{parts}

\question %03
\textbf{If we decreased the variance of $u_{i}$ by including more explanatory variables}, should we expect that the standard error on $\hat{\beta}_{1}$ will increase or decrease? Why?
\vspace*{\stretch{1}}

\end{questions}

\end{document}